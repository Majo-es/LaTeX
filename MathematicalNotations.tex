% Tutorial from FreeCodeCamp
\documentclass[11 pt]{article}
\pagestyle{empty} % turn off the pg #s
\usepackage{amsmath, amssymb, amsfonts}

\begin{document}

\textbf{Superscripts:} $$ 2x^3$$
% WITHOUT CURLY BRACKETS IT WOULD BE: 2 to the power of 3 and then 4 
$$2x^{34}$$
$$2x^{3x+4}$$
$$2x^{3x^4+5} $$

\textbf{Subscripts:}
$$x_1$$
$$x_{12}$$
$$x_{1_2}$$
$$x_{1_{2_3}}$$
$$a_0,a_1,a_2,\ldots,a_{100}$$

\textbf{Greek Letters:}
$$\pi$$ %lowercase pi
$$\Pi$$ %uppercase pi
$$\alpha$$
$$A=\pi r^2$$

\textbf{Trig Functions:}
$$y=\sin x$$
$$y=\cos x$$
$$y=\csc \theta$$
$$y=\sin^{-1} x$$
$$y=\arcsin x$$

\textbf{Log Functions:}
$$y=\log x$$
$$y=\log_5 x$$
$$y=\ln x$$

\textbf{Roots:}
$$\sqrt{2}$$
$$\sqrt[3]{2}$$
$$\sqrt{x^2+y^2}$$
$$\sqrt{ 1+\sqrt{x} }$$

\textbf{Fractions:}
$$\frac{2}{3}$$
About $\displaystyle \frac{2}{3}$ of the glass is full. \\[16pt]
About $\frac{2}{3}$ of the glass is full. \\[6pt]
% d -> display math mode 
About $\dfrac{2}{3}$ of the glass is full. \\[6pt]
$$\frac{\sqrt{x+1}}{\sqrt{x+2}} $$ \\[6pt]
$$\frac{1}{1+\frac{1}{x}}$$

\end{document}